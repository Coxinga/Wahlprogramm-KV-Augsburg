\chapter{Datenschutz und Transparenz}

  \section{Datenherausgabe durch Bürgerämter nur nach Zustimmung}
  
  Privatpersonen, Firmen, Kirchen, Parteien und andere Einrichtungen fordern 
  von Bürgerämtern gegen geringe Gebühren Daten über Bürger ohne deren 
  Einwilligung an, um diese zu privaten oder kommerziellen Zwecken zu 
  verwenden. Eine Weitergabe von Informationen über Bürger ohne deren 
  Einwilligung lehnen wir ab. Alle Bürgerämter der Stadt Augsburg sowie des 
  Landkreises werden angehalten, diese Praxis zu beenden, da sie dem 
  Grundrecht auf Informationelle Selbstbestimmung widerspricht. Stattdessen 
  muss in Zukunft sichergestellt sein, dass die Erlaubnis der Bürger eingeholt 
  wurde, bevor Informationen über sie herausgegeben werden. Wurde diese 
  Erlaubnis erteilt, soll der Bürger auf Anfrage Informationen über die 
  getätigten Abfragen erhalten und seine Erlaubnis jederzeit widerrufen 
  können. 
  
  \section{Informationsfreiheitssatzung für Augsburg}
  
  Die von der Bayerischen Staatsregierung und den Landtagsfraktionen der CSU 
  und der FDP eingenommene Ablehnungshaltung ist nicht hinnehmbar. Sie macht 
  deutlich, dass diese kein Interesse an der Informationsfreiheit der 
  Bayerischen Bürger haben, sondern in monarchischem Habitus ihr 
  Herrschaftswissen nach Gutdünken preisgeben wollen. Gerade die Vorgänge um 
  die Lebensmittelskandale der vergangenen Jahre machen deutlich, dass ein 
  Rechtsanspruch auf Information unverzichtbar ist. Auch vor dem Hintergrund, 
  dass auf Bundesebene und in zahlreichen Bundesländern die 
  Informationsfreiheit eingeführt wurde, ist es inakzeptabel, dies den 
  Bayerischen Bürgern auf Landesebene zu verwehren. In Nordrhein-Westfalen 
  gilt das Informationsfreiheitsgesetz schon seit einigen Jahren. Die dort 
  gemachten Erfahrungen zeigen, dass entgegen der Ansicht der übermächtigen 
  CSU-Vorbeter und der willfährigen FDP sehr wohl ein Bedürfnis der Bürger auf 
  mehr Information und Offenheit besteht. Die dortige Regelung ist 
  kostenneutral, weil für die Akteneinsicht angemessene Gebühren fällig 
  werden. Eine befürchtete Überlastung der Behörden ist nicht eingetreten. Es 
  spricht also aus Sicht des freien und mündigen Bürgers rein gar nichts 
  dagegen, der Bevölkerung einen Rechtsanspru"-ch auf Information durch die 
  Behörden einzuräumen.
  
  Die Piratenpartei Augsburg fordert nachdrücklich die Schaffung einer 
  gesetzlichen Grundlage zur Erleichterung der Information der Augsburger 
  Bürger durch die kommunalen Behörden. Es ist für freie und mündige Bürger 
  nicht hinnehmbar, dass ihnen die Einsicht in nicht geheimhaltungsbedürftige 
  behördliche Akten verwehrt wird. Nicht die Bürger müssen darlegen, dass sie 
  ein besonderes Interesse an der Einsicht in behördliche Akten und Vorgänge 
  haben, sondern der Staat muss darlegen und nachweisen, weshalb er seinen 
  Bürgern die Einsicht verwehren will. Die Bürger müssen einen Rechtsanspruch 
  auf behördliche Informationen bekommen. Das Informationsfreiheitsgesetz des 
  Landes Nordrhein-Westfalen (NRW) kann hierzu als Anhaltspunkt dienen.
  
  \section{Stadtwerke GmbH wieder in einen Eigenbetrieb der Stadt Augsburg 
  überführen, um Transparenz wiederherzustellen}
  
  Die Piratenpartei Augsburg setzt sich dafür ein, die Stadtwerke Gm"-bH 
  wieder 
  in einen Eigenbetrieb der Stadt Augsburg zu überführen, um Transparenz 
  wiederherzustellen. Diese Maßnahme erlaubt, zusammen mit einer 
  Informationsfreiheitssatzung für die Stadt Augsburg, die Kontrolle der 
  Stadtwerke durch die Bürger. Momentan sind Auskünfte von den insgesamt 8 (!) 
  GmbHs aus denen die Stadtwerke Augsburg bestehen mit Verweis darauf 
  verweigerbar, dass es sich um Geschäftsgeheimnisse handle. Eine Struktur aus 
  einer   Holding mit diversen separat firmierenden Gliederungen ist alles Andere 
  als Transparent.
  
  \section{Keine Videoüberwachung im öffentlichen Raum}
  
  Die Beobachtung und Überwachung des öffentlichen Raums lehnen wir strikt ab. 
  Und dies unabhängig davon, ob die Überwachungsmaßnahmen durch private oder 
  öffentliche Hand betrieben werden. Bereits angewandte Maßnahmen und der 
  Einsatz neuer Technologien bei der öffentlichen Überwachung sind kritisch zu 
  hinterfragen und gegebenenfalls rückgängig zu machen. Das Gefühl, durch 
  Kameras beobachtet zu werden, vermittelt keine Sicherheit, sondern schränkt 
  die persönliche Freiheit ein.
  
  Während Videoüberwachung immer weiter verbreitet ist, wird bei 
  Präventionsarbeit und städtebaulichen Maßnahmen gespart, die Ihre 
  langfristige Sicherheit verbessern würden.
  
  Deshalb wollen wir, dass bei derzeitig bestehenden Videoüberwachungen 
  systematisch durch die kommunale Verwaltung Augsburgs überprüft wird, ob sie 
  erforderlich, geeignet und verhältnismäßig sind. Genügen bestehende oder 
  geplante Maßnahmen nicht den gesetzlichen Vorgaben, sind sie abzulehnen. 
  Eine anlasslose Überwachung von Menschen durch Kameras ist ein Eingriff in 
  das allgemeine Persönlichkeitsrecht und das Recht auf informationelle 
  Selbstbestimmung. 
  
  \section{Keine Videoüberwachung in öffentlichen Verkehrsmitteln}
  
  Die Piratenpartei Augsburg lehnt Videoüberwachung in öffentlichen 
  Verkehrsmitteln ab, weil sie Ausdruck eines pauschalen Verdachts gegenüber 
  Fahrgästen, Fahrzeugführern oder Begleitern des Verkehrsmittels ist. Oftmals 
  wird vergessen, dass neben den Fahrgästen natürlich auch Betriebsangestellte,
  wie Busfahrer oder Bahnschaffner, von der ständigen Überwachung betroffen 
  sind. Dies lässt sich mit einem verantwortungs"-bewussten 
  Arbeit"-nehmer"-daten"-schutz nicht vereinbaren.
  
  Städtische Betriebe oder öffentlich beauftragte Dienstleistungsunternehmen 
  sollen sich von derartigen anlasslosen Maßnahmen distanzieren und bereits 
  installierte Videokameras entfernen. Die zur rein technischen Bedienung der 
  Fahrzeuge notwendige Kameras bleiben davon unberührt.
  
  Mit Unternehmen, die diese Bedingungen in den im öffentlichen Nahverkehr 
  Augsburgs eingesetzten Bussen und Bahnen nicht erfüllen, sollen zukünftig 
  keine Verträge mehr geschlossen werden. Kameras helfen niemandem in 
  gefährlichen Situationen. Stattdessen kann Gewalt tatsächlich nur verhindert 
  werden, indem das Begleitpersonal aufgestockt wird. Dies wollen wir umsetzen 
  und damit für mehr echte Sicherheit sorgen.

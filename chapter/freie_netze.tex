\chapter{Offene Standards und Freie Netze}
  
  \section{Freie Netzte}
  
  In vielen Städten und Gemeinden, auch in Augsburg, gibt es lokale, von 
  einzelnen Bürgern betriebene und selbstverwaltete (Funk-)Netzwerke. Diese 
  fördern die lokale Kommunikation der Bürger untereinander und helfen mit, 
  digitale und soziale Gräben zu überwinden. Beispielsweise kann ein solches 
  Funknetz die Infrastruktur für eine stadtweite Basisversorgung mit Internet 
  via WLAN bilden. Eine wichtige Voraussetzung für den Erfolg solcher freien 
  Netzwerke ist es, Zugang zu funktechnisch optimalen Standorten zu haben, auf 
  denen Infrastruktur installiert wird. Hierdurch verbessert sich die 
  Netzabdeckung und mehr Menschen wird die Teilnahme an den Bürgernetzen 
  ermöglicht. Initiativen wie z.B. freifunk.net, die sich den Aufbau freier 
  Netze zum Ziel gesetzt haben, scheitern jedoch häufig an bestehenden 
  Beschränkungen bei Zugang zu solchen optimalen Standorten. Durch die 
  Kooperation der Stadt Augsburg können hier jedoch weitere attraktive 
  Standorte erschlossen werden.
  
  Der Kreisverband Augsburg der Piratenpartei fordert den Auf- und Ausbau 
  freier Funknetze und die Unterstützung lokaler Initiativen wie z.B. 
  Freifunk. Die Stadtregierung gewährt den Initiatoren Zugang zu 
  städtischen Gebäuden, um die günstige Lage der zentral gelegenen Gebäude zur 
  Erweiterung zu nutzen. 
\chapter{Offene Standards und Freie Netze}
  
  \section{Freie Netze}
  
  In vielen Städten und Gemeinden, auch in Augsburg, gibt es lokale, von 
  einzelnen Bürgern betriebene und selbstverwaltete (Funk-)Netzwerke. Diese 
  fördern die lokale Kommunikation der Bürger untereinander und helfen mit, 
  digitale und soziale Gräben zu überwinden. Beispielsweise kann ein solches 
  Funknetz die Infrastruktur für eine stadtweite Basisversorgung mit Internet 
  via WLAN bilden. Eine wichtige Voraussetzung für den Erfolg solcher freien 
  Netzwerke ist es, Zugang zu funktechnisch optimalen Standorten zu haben, auf 
  denen Infrastruktur installiert wird. Hierdurch verbessert sich die 
  Netzabdeckung und mehr Menschen wird die Teilnahme an den Bürgernetzen 
  ermöglicht. Initiativen wie z.B. freifunk.net, die sich den Aufbau freier 
  Netze zum Ziel gesetzt haben, scheitern jedoch häufig an bestehenden 
  Beschränkungen bei Zugang zu solchen optimalen Standorten. Durch die 
  Kooperation der Stadt Augsburg können hier jedoch weitere attraktive 
  Standorte erschlossen werden.
  
  Der Kreisverband Augsburg der Piratenpartei fordert den Auf- und Ausbau 
  freier Funknetze und die Unterstützung lokaler Initiativen wie z.B. 
  Freifunk. Die Stadtregierung gewährt den Initiatoren Zugang zu 
  städtischen Gebäuden, um die günstige Lage der zentral gelegenen Gebäude zur 
  Erweiterung zu nutzen. 
  
  \section{Freie Software in kommunaler IT-Infrastruktur}
  \label{sec:Freie Software in kommunaler IT-Infrastruktur}
  
  Die Piratenpartei Augsburg setzt sich für die Umstellung der IT-Landschaft 
  auf freie Software ein. Der konsequente Einsatz offener Software reduziert 
  die laufenden IT-Infrastrukturkosten in erheblichem Maß durch den Wegfall 
  der Lizenzkosten proprietärer Software:
  
  \begin{itemize}
    \item Durch die Einführung von quelloffener Software und die Benutzung 
    offener Standards werden nicht nur Kosten eingespart, sondern auch die 
    regionale Wirtschaft gestärkt.
    \item Kommunikation mit dem Bürger sollte vollständig in offenen Formaten 
    erfolgen.
    \item Innerhalb der Verwaltung soll die Kommunikation ebenfalls offene 
    Standards nutzen, sofern dem keine schwerwiegenden Hindernisse 
    entgegenstehen.
    \item Die Kommune Augsburg profitiert in der öffentlichen Wahrnehmung, wie 
    auch München, durch eine technologische Vorreiterrolle. 
  \end{itemize}
  
  Proprietäre und kommerzielle Software bietet keine Garantie dafür, dass der 
  Benutzer die Inhalte, die er heute produziert, auch morgen noch abrufen 
  kann. Ebenso können damit erstellte Dokumente von Bürgern unter Umständen 
  erst nach dem Erwerb kostenpflichtiger Software legal genutzt werden.
  
  Aus diesem Grund haben sich die Europäische Kommission, das Bundesamt für 
  Sicherheit in der Informationstechnologie, die Stadtverwaltung Bristol, die 
  Stadtverwaltung München, die Stadtverwaltung Wien, die französische 
  Gendarmerie, diverse Ministerien und staatliche Einrichtungen in Indien 
  (darunter die National Bank und der Oberste Gerichtshof) sowie die 
  Unternehmen Oracle, IBM, Lenovo und Orange (Mobilfunk) für den Einsatz von 
  OpenOffice.org und damit auch für unabhängige Formate und offene Standards 
  in der Verwaltung entschieden. Der Wechsel ist mittlerweile durchaus 
  komfortabel möglich, in der Stadtverwaltung München arbeiten bereits 15.000 
  PCs mit diesem System; dadurch konnten 33\% der IT-Kosten eingespart werden. 
  Für die Umstellung fallen im wesentlichen Kosten in Form von Schulungskosten 
  an, diese fallen jedoch bei Updates auf neue Versionen anderer Programme 
  ebenfalls an.
  
  Die Stadt Augsburg würde sich damit in eine Riege erfolgreicher Verwaltungen 
  einreihen, die mit dem Wechsel zu offenen Standards und freier Software die 
  Verwaltungskosten gesenkt haben, die Region durch Förderung regionaler 
  IT-Projekte stärken und gleichzeitig die politische Teilhabe der Bürger 
  erleichtern würde.
  
  \section{Offene Standards in kommunaler Verwaltung und Politik}
  
  Politik und Verwaltung sollen transparente und kostengünstige Einrichtungen 
  zur Information und Organisation des öffentlichen Lebens für den Bürger sein 
  und nicht nur dessen behördeninterne Verwaltung sicher stellen. Die 
  Piratenpartei fordert daher die leichte Zugänglichkeit aller öffentlicher 
  Unterlagen durch Verwendung offener Standards, sofern diese Unterlagen nicht 
  die berechtigten Datenschutzinteressen des Einzelnen gefährden. Offene 
  Standards wie ODT für Dokumente oder SIP für Internettelefonie sind ohne 
  Einschränkungen für jeden Bürger zugänglich und steigern dadurch die 
  Teilhabe am politischen Geschehen. Der Einsatz von offenen Standards und 
  Formaten bietet außerdem die Möglichkeit plattformunabhängiger 
  Ausschreibungen für die IT-Infrastruktur. So können auch kleine und 
  mittelständische Händler aussichtsreich an Ausschreibungen der öffentlichen 
  Hand teilnehmen, was zu einer Regionalisierung der Gewinnschöpfung und 
  Stärkung eines lokalen Wirtschafts- und Technologiestandorts führt.
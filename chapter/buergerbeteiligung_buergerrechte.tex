\chapter{Bürgerbeteiligung und Bürgerrechte}
  
  \section{Beteiligung der Bürger an Projekten und Großprojekten}
  
  Die jüngere Geschichte Augsburgs ist durchzogen von intransparenten und 
  jahrelang umstrittenen Bauvorhaben und Großprojekten. Die Stadtregierung hat 
  hier immer wieder bewiesen, wie durch mangelhafte Bürgerbeteiligung wichtige 
  Projekte gefährdet werden können. Bürgerbeteiligung beginnt mit Information 
  und Transparenz. Akzeptanz von Projekten wird durch Nachvollziehbarkeit 
  geschaffen. Deshalb muss über geplante Projekte von Anfang an umfassend 
  informiert werden.
  
  Birgt ein Projekt viel Streit-Potenzial, sollen die Bürger stärker beteiligt 
  werden, zum Beispiel durch Maßnahmen wie Diskussionsveranstaltungen, 
  Bürgerbefragungen bis hin zu bindenen Bürgerentscheiden.
  
  \pagebreak
  \section{Einführung eines Bürgerhaushalts}

  \subsection{Grundverständnis}
  
  Ein Bürgerhaushalt ermöglicht eine Mitarbeit der Bürger am Haushaltsplan und 
  damit auch ein Mitbestimmen über die Verwendung von kommunalen 
  Finanzmitteln. Bürgerhaushalte beschränken sich nicht auf einzelne 
  Stadtteile, sondern sind für die gesamte Stadt bzw. Gemeinde ausgelegt und 
  als dauerhafte, regelmäßig wiederkehrende Verfahren angelegt.
  
  Beispiele wie der Stuttgarter Bürgerhaushalt zeigen, dass ein nur auf 
  Vorschläge zielendes Verfahren den Ansprüchen eines richtigen 
  Bürgerhaushaltes nicht genügt. Statt der Erstellung eines wilden 
  Wunschzettels, der nicht weiter beachtet werden muss, sollen die Bürger über 
  die Verwendung von einem vorbestimmten Teil der Finanzen mitbestimmen. Der 
  Stadt- bzw. Gemeinderat hat das letzte Wort und muss die Vorschläge bzw. 
  Prioritäten auf Gültigkeit und Umsetzbarkeit prüfen und bewerten, ist dann 
  aber an die Umsetzung gebunden und gegenüber den Bürgern 
  rechenschaftspflichtig.
  
  Deshalb setzt sich die Piratenpartei Augsburg für die Ein- und Durchführung 
  von dauerhaften Bürgerhaushalten ein.
  
  \subsection{Transparenz}
  
  Transparenz ist Voraussetzung, Begleiter und Ziel eines Bürgerhaushaltes. Es 
  muss von Anfang an verständlich zugänglich sein, woher die Stadt oder 
  Gemeinde ihr Geld bekommt und was damit gemacht werden soll. Die Prozesse 
  der Haushaltsplanung müssen für die Bürger nachvollziehbar sein.
  
  \subsection{Information}
  
  Deshalb ist eine intensive Informationsphase und Öffentlichkeitsarbeit 
  wichtig. Maßnahmen zur Information sind u.a. Flyer und Broschüren, 
  Informationsveranstaltungen, Zeitungsanzeigen und das Internet. Ein 
  Bürgerhaushalt benötigt die Unterstützung des Gemeinderates und der 
  Verwaltung.
  
  \subsection{Dialog- und Beteiligungsphase}
  
  Die Art der Beteiligung kann vielfältig sein und muss auf die Anforderungen 
  und Spezifikationen der einzelnen Kommune angepasst sein. Möglich sind zum 
  Beispiel Prioritätenlisten für Investitionen oder Einsparungen, denen die 
  Bürger ihre Stimme geben oder das Einbringen von eigenen Vorschlägen, die 
  bindend in den Haus"-halts"-plan einfließen. Diskussions"-möglichkeiten 
  müssen ge"-sch"-affen werden. Gemeinderatsmitglieder und 
  Verwaltungs"-mitarbeiter sollten sich an der Diskussion beteiligen. Das 
  stärkt den Dialog und das so eingebrachte Fachwissen kann eine Hilfe sein. 
  Das Internet bietet sich als Plattform für die Durchführung und Diskussion 
  an, moderiert und und idealerweise ergänzt durch von der Gemeinde 
  organisierte Veranstaltungen. Aber auch eine Durchführung auf Papier, zum 
  Beispiel zusammen mit den Gemeindebriefen verteilt, ist denkbar. Wird der 
  Bürgerhaushalt online durchgeführt, muss technisches Know-how vorhanden 
  sein, um einen Missbrauch so gering wie möglich zu halten. Dies ist auch mit 
  beschränkten finanziellen Mitteln möglich.
  
  \subsection{Rechenschaft}
  
  Rechenschaft ist schließlich ein wichtiger Faktor. Der Stadt- sowie die 
  Gemeinderäte muss transparent darlegen, was umgesetzt und was abgelehnt 
  wurde und dies begründen. Eine transparente Analyse und Darstellung der 
  Folgen und Ergebnisse der Bürgerentscheidungen bilden den Auftakt des 
  nächsten Bürgerhaushaltes für den kommenden Haushaltsplan.
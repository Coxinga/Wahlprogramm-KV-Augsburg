\chapter{Bildung}
  
  \section{Ganztagesbetreuung ausbauen}
  
  Berufliche und familiäre Lebensumstände machen eine umfassende Betreuung an 
  Schulen über die reine Wissensvermittlung hinaus notwendig. Die 
  Piratenpartei Augsburg setzt sich dafür ein, das bereits bestehende Angebot 
  an Ganztagsbetreung bedarfsorientiert auszubauen und dabei hohe pädagogische 
  und räumliche Qualitätsstandards zu gewährleisten.
  
  \section{Inklusion ermöglichen und erleichtern}
  
  Die Piratenpartei Augsburg setzt sich dafür ein, durch die Schaffung von 
  zusätzlichen Personalstellen die Inklusion von Kindern mit geistigen oder 
  körperlichen Einschränkungen zu ermöglichen. Des Weiteren sollen diese 
  Bemühungen, um ein ganzheitliche Gesellschaft, durch einfache aber effektive 
  bauliche Maßnahmen weiter erleichtert werden. Langfristiges Ziel ist die 
  Schaffung eines barrierefreien Zugangs zu allen städtischen 
  Bildungseinrichtungen.
  
  \section{IT-Ausstattung der Schulen verbessern}
  
  Unter Mitarbeit von pädagogischen Fachkräften soll ein modernes und 
  zukunftsorientiertes Lernen an Schulen auch im IT-Bereich ermöglicht werden. 
  Häufig ist die vorhandene Hardware-Ausstattung in Schulen einige Jahre alt 
  und kann somit nicht mehr mit modernen Betriebssystemen und aktueller 
  Software genutzt werden, da für ältere Geräte keine Treiber bereitgestellt 
  und Programme entwickelt werden. Diese bereits bestehende Infrastruktur 
  soll, auch in Anlehnung an die Umstellung auf freie Software in der 
  Kommunalverwaltung\footnote{ToDo: Verlinkung auf den Programmpunkt}, in die 
  schrittweise Modernisierung eingebunden werden. Die Verwendung freier 
  Software ermöglicht es hierbei, kostengünstig existierende Ausstattung 
  weiter zu verwenden und dennoch hochwertige Bildungsarbeit zu leisten.
  
  \section{Sanierung und Modernisierung der Gebäude}
  
  Die räumlichen Bedingungen an vielen Schulen in Augsburg sind katastrophal. 
  Durch ein ausgereiftes und umfassendes Sanierungskonzept soll dieser Zustand 
  nachhaltig behoben werden. Das Konzept wird in Zusammenarbeit mit Experten 
  erstellt und durchgeführt, speziell unter dem Gesichtspunkt der Energie- und 
  Kosteneffizienz. Die flächendeckende Sanierung soll nicht nur bauliche 
  Verbesserungen, sondern auch personelle Aufstockung im Bereich des Wartungs- 
  und Reinigungspersonals beinhalten. Durch zusätzliche Arbeitskräfte soll 
  nicht nur die Sauberkeit gewährleistet, sondern auch das bereits vorhandene 
  Personal entlastet werden.
  
  \section{Sozialpädagogische Arbeit stärken}
  
  Unterstützend zum Ausbau der Ganztagesbetreuung soll durch zusätzliches 
  Fachpersonal und hochwertige räumliche Ausstattung die sozialpädagogische 
  Arbeit an den Schulen erweitert und verbessert werden. Die dadurch 
  entstehenden Arbeitsplätze sind ein weiterer Pluspunkt für den Lebensraum 
  Augsburg.
  
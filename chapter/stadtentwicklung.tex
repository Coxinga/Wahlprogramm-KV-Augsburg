\chapter{Stadtentwicklung}

  \section{Echten öffentlichen Raum erhalten - gegen den temporären und 
  permanenten Verlust öffentlichen Raumes}
  
  Die Piratenpartei setzt sich für Aufenthaltsqualität und eine Belebung des 
  öffentlichem Raumes ein. Dazu gehöhrt unter anderem die freie Nutzbarkeit 
  der Flächen durch jeden Bürger ohne Konsumzwang. Die Piratenpartei sieht 
  darin eine Umsetzungsvariante der gesellschaftlichen Teilhabe und eine 
  Stärkung des lokalen gesellschaftlichen Lebens. Für die Bevölkerung der 
  Stadt bedeutet diese Veränderung, eine Steigerung der Lebensqualität, da die 
  zentralen Orte nicht mehr nur Konsumraum sind, sondern auch Raum zum 
  zwanglosen verweilen und Kommunikation jenseits der eigenen 
  Konsummöglichkeiten ist.
  
  Die Praxis der letzten Jahre, Teile des öffentlichen Raumes als 
  Gastronomieaußenfläche zu verpachten, steht somit den Zielen der 
  Piratenpartei entgegen. Hierdruch wird aus öffentlichen Raum ein Raum 
  gemacht, in dem private Rechte, beispielsweise ein Konsumzwang, durchsetzbar 
  sind.
  
  Anstelle der Verpachtung als Gastronomieaußenfläche sollen die 
  entsprechenden Gremien die Fläche als Aufenthaltsfläche gestalten. Auf einer 
  solchen soll man genauso die gastronomischen Leistungen in Anspruch nehmen 
  können, jedoch sich auch so treffen können.
  
  Kommunen sollen über eine Änderung des Kommunalen Abgabengesetzes die 
  Möglichkeit gegeben werden, die Flächen zu gestalten. Damit wird verhindert, 
  dass die Gastronomen die Flächen mit Außenmöbeln gestalten, die schrille 
  Werbeaussagen für Kaffee- oder Biermarken tragen. So wird auch das Stadtbild 
  aufgewertet.
  
  \section{Instandhaltung und Sanierung von Bauwerken und Infrastruktur im 
  Stadtgebiet}
  
  Die Bausubstanz vieler öffentlicher Bauwerke im Stadtgebiet ist in einem 
  sanierungsbedürftigen Zustand. Obwohl in der Vergangenheit teilweise 
  entsprechende Prüfungen der Bausubstanz durchgeführt wurden, erfolgte - 
  abgesehen von vereinzelten energetischen Sanierungsmaßnahmen - keine 
  zeitnahe Instandsetzung der Bauwerke.
  
  Durch eine verzögerte Mängelbeseitigung entstehen jedoch weitere Schäden, so 
  dass die vordergründige Kostenersparnis durch einen erhöhten und 
  kostenintensiveren Instandsetzungsbedarf zu einem späteren Zeitpunkt erkauft 
  wird. Ab einem entsprechenden Schädigungsgrad, ist eine wirtschaftliche 
  Sanierung überhaupt nicht mehr möglich, so dass ein teuer Abriss und Neubau 
  des Bauwerks erfolgen muss. Dies bedeutet im Endeffekt eine Verschwendung 
  von Steuergeldern.
  
  Zusätzlich können durch mangelhafte Prüfung und Wartung von Bauteilen und 
  Gebäuden erhebliche Sicherheitsrisiken entstehen, welche zu Personenschäden 
  führen können. Katastrophen wie in Bad Reichenhall, die Menschenleben 
  fordern, müssen unbedingt vermieden werden.
  
  Die Piratenpartei setzt sich aus diesem Grund dafür ein, die Bausubstanz 
  aller öffentlichen Bauwerke in regelmäßigen Abständen durch internes oder 
  externes, qualifiziertes Fachpersonal überprüfen zu lassen. 
  Instandsetzungsbedürftige Bauwerke sollen nach Möglichkeit kurzfristig 
  instandgesetzt werden, um höhere Kosten durch Folgeschäden zu vermeiden. 
  Hierfür müssen im städtischen Haushalt entsprechende Mittel eingeplant und 
  den Bauämtern bereitgestellt werden.
  
  \section{Königsplatz / Haltestellen wieder zu echtem öffentlichen Raum 
  machen}
  
  Momentan sind der Königsplatz und die anderen Haltestellen in Augsburg durch 
  die Ausgliederung des öffentlichen Nahverkehrs, als Stadtwerke Augsburg 
  Verkehrs-GmbH halböffentlicher Raum, in dem Sicherheitsdienste im Rahmen des 
  Hausrechts Zutritt verwehren, oder Menschen verweisen können.
  
  Verkehrsknotenpunkte sollten stets öffentlicher Raum sein. Dass heißt für 
  jeden Menschen gleichermaßen zugänglich. Für die öffentlichen Ordnung sorgt 
  dann die Polizei, statt wenig vertrauenerweckende private
  Sicherheitsdienste. Die Polizei kann zwar auch einen Platzverweis unter 
  bestimmten gesetzlichen Voraussetzungen erteilen, aber nicht willkürlich, 
  wie es das Hausrecht der Stadtwerke GmbH erlaubt.
  
  \section{Öffentliche Wasserspender im Stadtgebiet}
  
  Wasser ist ein Menschenrecht. Daher sollte Trinkwasser allen Menschen, auch 
  wenn diese gerade unterwegs sind, kostenfrei zur Verfügung stehen. Zur 
  Einlösung dieser Forderung sollen an stark frequentierten öffentlichen Orten 
  im Stadtgebiet (beispielsweise, aber nicht nur an Verkehrknotenpunkten, sondern auch in 
  der Fussgängerzone und an Schulen) Wasserspender aufgestellt werden.
    
  \begin{itemize}
    \item Eine ausreichende Hydrierung kann besonders im Sommer helfen 
          Kreislaufbeschwerden, sowie Kopfschmerzen und daher Stress für den 
          Einzelnen, aber auch bei Anderen vorzubeugen.
    \item Niemand wäre mehr gezwungen seinen Durst später zu befriedigen, wenn
          dies sofort aufgrund der verfügbaren finanziellen Mittel, oder 
          verfügbarer Zeit nicht möglich ist, entgeltlich und ggf mit Wartezeit 
          ein Getränk zu erwerben.
    \item Der überall kostenfreie zugängliche Durstlöscher aus dem 
          Trinkwasserspender hilft mit, die möglicherweise ungesunde Vorliebe 
          von Kindern und Jugendlichen für Softdrinks zu verringern. 
  \end{itemize}
  
  Neben den gesundheitlichen Vorteilen bedeutet ein geringerer Konsum an meist 
  PET-Flaschen mit Wasser, oder Softdrinks auch weniger Ressourcenverbrauch für 
  Verpackung und Transport von Getränken.
  
  Letztlich befördert diese Maßnahme auch die touristische Attraktiviät der 
  Stadt, da dem Gast vermittelt wird, dass er hier nicht zur Befriedigung 
  eines so grundlegenden Bedürfnisses, wie Durst zu Kasse gebeten wird.
  
  Als Beispiel kann die bisherige Praxis der Stadt Wien heran gezogen 
  werden.\footnote{\url{http://www.vienna.at/mobile-trinkbrunnen-wiener-hochquellwasser-gegen-durst/news-20110504-11013289}}
   

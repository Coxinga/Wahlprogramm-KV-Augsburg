\chapter{Infrastruktur}

  \section{Barrierefreiheit}
  
  Die Stadt Augsburg verpflichtet sich zur Einhaltung der UN-Konvention über 
  die Rechte von Menschen mit Behinderungen. Barrierefreiheit ist bei allen 
  städtischen Projekten bereits zu Beginn in den Planungsphasen zu 
  berücksichtigen. Alle Kreuzungen und Haltestellen sind zeitnah barrierefrei 
  mit abgesenkten Bordsteinen und taktilen Flächen für Blinde und 
  sehbehinderte Menschen zu versehen, Fußgängerampeln werden mit Audiosignalen 
  nachgerüstet und regelmäßig gewartet. Die Innenstadt ist mittelfristig nach 
  dem Vorbild anderer Kommunen mit einem Leitsystem für Blinde und 
  Sehbehinderte auszustatten. Das Leitsystem soll in enger Zusammenarbeit mit 
  den Behindertenverbänden, sowie unter Koordination durch den 
  Behindertenbeirat der Stadt Augsburg erarbeitet und auf die örtlichen 
  Gegebenheiten angepasst werden.
  
  \section{Barrierefreie Spielplätze fördern}
  
  Ein barrierefreier Spielplatz muss die ganze Vielfalt aller Menschen 
  abdecken und ist grundsätzlich nicht sonderlich teurer als nicht 
  Barrierefreie. Man muss die Projekte nur von Anfang an richtig planen und 
  durchdenken. Spielplätze sind Begegnungsorte. Hier treffen sich Menschen 
  unterschiedlichen Alters, aus verschiedenen Gesellschaftsschichten, Menschen 
  mit und ohne Behinderung.
    
  Die Piratenpartei Augsburg setzt sich dafür ein, barrierefreie Spielplätze 
  im Stadtgebiet zu fördern und zu fordern. Bereits bestehende Einrichtungen 
  sollen im Sinne der Barrierefreiheit und Inklusion sukzessive erweitert bzw. 
  saniert werden. Spielplätze für Rollstuhlfahrer, Spielgeräte die für 
  Rollstuhlfahrer nutzbar sind und Beschäftigungsmöglichkeiten, die von einem 
  Rollstuhl aus durchgeführt werden können, sollen hierbei geschaffen werden. 
  Für blinde oder sehbehinderte Spielplatzbesucher sollen Orientierungshilfen,
  wie Leitlinien oder auffallend farblich gekennzeichnete Bereiche geschaffen 
  werden. Auf Geräten soll der Gleichgewichtssinn beansprucht und geschult 
  werden können. Anregungen für den Geruchssinn, den Hörsinn oder Tastsinn 
  sollen geschaffen werden. Soweit wie möglich sollen Spielplätze eine 
  nahegelegene (behindertengerechte) Toilette oder gar eine Wickelmöglichkeit 
  bieten. Ruhe- und Schattenplätze wären weiterhin erstrebenswert.
  
  \section{Bessere Vernetzung von Fahrradwegen im ländlichen Raum}
  
  Inzwischen sind im Landkreis Augsburg etliche Fahrradweg entlang der 
  Ortsverbindungsstraßen entstanden. Leider enden diese Fahrradwege oft direkt 
  am Ortsanfang oder kurz danach, ohne die Radfahrer wieder in den allgemeinen 
  Verkehrsfluss zu integrieren. Typisches Beispiel sind etwa gemeinsam 
  genutzte Rad- und Fußwege, die auf einmal zu reinen Fußwegen umgewidmet 
  werden, ohne dass diese eine Möglichkeit zur Ausleitung auf die Straße 
  erhalten. Radfahrer machen einen großen Teil des aktiven Verkehrs innerhalb 
  von Ortschaften aus. Trotzdem sind sie bisher kaum sinnvoll in bestehende 
  Verkehrskonzepte eingebunden.
  
  Die Piratenpartei setzt sich dafür ein, dass Fahrradwege durch Gemeinden 
  ausgewiesen und gebaut werden. Wenn eine Routenführung der Durchgangsstraße 
  entlang nicht möglich ist, sollten Alternativrouten über Seitenstraßen oder 
  um die Gemeinde herum entsprechend ausgebaut und beschildert werden.
  
  Vor allem bei Neu- und Umbauten von Ortsdurchfahrten muss auf einen 
  fahrradgerechten Ausbau geachtet werden. Hier hält sich die finanzielle 
  Mehrbelastung durch frühzeitige Planungsmöglichkeiten in Grenzen; es wird 
  zudem ein integriertes Verkehrskonzept ermöglicht, das verschiedenartige 
  Verkehrsteilnehmer wie Radfahrer, Fußgänger und Autofahrer gleichermaßen 
  berücksichtigt.
  
  Auch beim Neubau von Ortsumgehungsstraßen sollten zusätzlich Fahrradwege 
  entlang der Umgehung angelegt werden.
  
  \section{Öffentlicher Nahverkehr}
  
  Der Öffentliche Personennahverker (ÖPNV) spielt eine zentrale Rolle für die 
  Entwicklung einer Stadt und für die Versorgungsqualität der städtischen 
  Einrichtungen für die Bürger. Augsburg mit seiner dichtbebauten, kompakten 
  und strassenarmen Innenstadt ist als Versorgungsmittelpunkt Nordschwabens 
  anhaltend verkehrsüberlastet.
  
  Auch die kostspieligen und kontrovers aufgenommenen Strassenbaumassnahmen 
  der letzten Jahre konnten den Individualverkehr nur teilweise neu ordnen, 
  ohne ihn als zentrales Mittel der Mobilität zu stark einzuschränken. Die 
  Kosten dieser und weiterer Strassenbaunahmen belasten zusätzlich den 
  städtischen Haushalt und schränkt die zukünftige Gestaltungsfähigkeit 
  Augsburgs ein. Mit Verkehrsregeln und Verboten allein kann man den Bürgern 
  nicht zu städteplanerischer Weitsicht bewegen: Man muss ihnen ein besseres 
  Angebot machen.
  
  Glücklicherweise gibt es mit dem ÖPNV als Lösungsansatz für innerstädtische 
  Verkehrsprobleme bereits umfassende Erfahrungen. Die grundsätzliche 
  Wichtigkeit eines öffentlichen Transportnetzes steht heute ausser Frage; um 
  es allerdings zur vollen Wirkung zu bringen, haben bereits mehrere Städte 
  erfolgreich die Finanzierung des Nahverkehrsnetzes neu definiert: Ebenso wie 
  Strassen, die dem Bürger kostenlos zur Verfügung stehen, entfaltet auch der 
  ÖPNV seine Wirkung als Instrument der Stadtentwicklung erst, wenn er 
  fahrscheinlos gemacht wurde. Fahrscheinlos bedeutet nicht kostenlos. Die 
  Finanzierung erfolgt über ein Umlageverfahren: Durch den Wegfall des 
  Vertriebs der Fahrscheine kommt es zu Einsparungen, während die signifikant 
  ansteigenden Umsätzen bei Gastronomie und Einzelhandel im Innenstadtbereich 
  mit mehr Einnahmen generieren.
  
  Die Piratenpartei Augsburg schlägt zur Finanzierung das Model 2 x 15 vor. 
  Für die Finanzierung des fahrscheinlosen ÖPNV wird eine Gebühr in Höhe von 
  15€ je Monat pro Einwohner erhoben. Schüler, Stundenten und soziale Schwache 
  sind von der Gebühr befreit. Die paritätische Beteiligung an der 
  Finanzierung der Unternehmen in Augsburg wird durch einen Beitrag von 15€ 
  pro Arbeitnehmer je Monat erreicht.
  
  Die Vorteile des fahrscheinlosen ÖPNV sind dagegen massiv:
  
  Wie schon in anderen Städten bringt ein für den Benutzer kostenfreier 
  Nahverkehr einen deutlichen Besucher-Zustrom in die Innenstadt: Handel und 
  Gastronomie blühen in Vergleichsszenarien um bis zu einem Drittel auf. Die 
  konstant verkehrsüberlastete Augsburger Innenstadt würde in grossem Umfang 
  immissions-entlastet. Sowohl Lärm als auch Abgase werden durch stärkere 
  Nutzung des Nahverkehrs drastisch reduziert, was dem Nachholbedarf Augsburgs 
  beim Umweltschutz sehr entgegenkommt. Die freie Mobilität via ÖPNV entlastet 
  die sozialen Spannungen in der von Immigration geprägten nordschwäbischen 
  Metropole; Schüler, Studenten, Auszubildende, Sozial Schwache und Senioren 
  erhalten durch den F-ÖPNV bessere Lebensqualität.
  
  Die Piratenpartei Augsburg will angesichts dieser eklantanten Vorteile den 
  fahrscheinlosen Öffentlichen Personennahverkehr in Augsburg einführen.
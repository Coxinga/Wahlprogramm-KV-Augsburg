\chapter{Kultur}
  
  % https://wiki.piratenpartei.de/BY:Augsburg/Kreisparteitag_2013.1/
  % Programm%C3%A4nderungsantr%C3%A4ge/PA501
  \section{Förderung der kulturellen Entwicklung von Nachwuchskünstlern}
  Augsburg ist nicht nur historisch gesehen ein Zentrum kultureller 
  Entwicklung. Auch in unserer Gegenwart gibt es bei uns ein großes Potential 
  an kunstschaffenden Menschen, sowohl im musischen wie auch im 
  gestalterischen Bereich.
  
  Die Förderung der Nachwuchskünstler beschränkt sich jedoch bislang 
  überwiegend auf die Aspekte der Ausbildung in Form von verschiedenen 
  (Fach-)Schulen oder auf einzelne, lokale Events, wie z.B. den 
  „Band-des-Jahres“-Wettbewerb. Im Alltag jedoch bietet Augsburg den 
  Nachwuchskünstlern nur wenig Möglichkeiten, sich zu entfalten bzw. ihre 
  Kunst zu präsentieren.
  
  Die Piratenpartei setzt sich aus diesem Grund für einen Ausbau der 
  Nachwuchsförderung künstlerisch ambitionierter Nachwuchstalente durch die 
  Schaffung geeigneter Rahmenbedingungen ein.\\
  \\
  Mögliche Vorschläge hierzu wären:
  
  \begin{itemize}
  \item Ausbau kostengünstiger Proberäume für Musikbands
  \item Organisation bzw. Bereitstellung von Flächen für die Aufführungen
        von Musik- oder Theaterproduktionen abseits etablierter Räume, z.B. 
        durch Festivals oder Auftrittsmöglichkeiten abseits von „Pay2Play“
  \item Freigabe von Flächen für Graffitigestaltungen
  \item Einrichtung von Ausstellungsflächen für bildnerische 
        Gestaltungskunst / Malereien in öffentlichen Gebäuden 
  \end{itemize}
  
  \section{Kulturpolitik}
  \subsection{Einleitung}
  
  Kultur ist ein wichtiges Gut, das man schützen und erhalten muss, sowie 
  einer der zentralen Standortfaktoren. In den letzten Jahren und Jahrzehnten 
  wurde in Augsburg Kulturförderung einseitig zugunsten von repräsentativer 
  Hochkultur betrieben. In Prestigeobjekte wie das Stadttheater, städtische 
  Museen oder die Festivals rund um Brecht oder Mozart wurden große Summen 
  investiert. Gleichzeitig fristen kleine, aber mit viel Einsatz geschaffene 
  Projekte ein Nischendasein. Selbst das überregional bekannte und 
  vielbeachtete lab30 musste mit massiven Budgetkürzungen und damit auch um 
  sein Überleben kämpfen.
  
  Trotz der beständigen Versuche, die sogenannte „Hochkultur“ zu fördern, 
  wird Augsburg von außen nur als kleiner Vorort und kulturell wenig 
  relevanter Nachbar Münchens wahrgenommen. Soll sich das ändern, müssen 
  Projekte gefördert und ins Leben gerufen werden, welche die Kulturszene für 
  ein lokales, aber auch überregionales Publikum interessanter machen. Dies 
  kann nur über die Stärkung von Nischenthemen sowie über eine 
  Spezialisierung in den einzelnen Kultursparten funktionieren.
  
  \subsection{Bedarfsanalyse}
  
  Um ein kommunales Kulturangebot realistisch planen zu können, sollte von 
  einer lokalen kulturwissenschaftlichen Fakultät untersucht werden, welche 
  Kulturformen von der Bevölkerung nachgefragt werden sowie welche benötigten 
  Kulturprojekte nicht ohne öffentliche Förderung existieren können.
  
  \subsection{Bildungsauftrag}
  
  Öffentlich finanzierte Kulturangebote sind darauf hin zu prüfen, ob sie 
  einen Bildungsauftrag gegenüber der Gesamtbevölkerung wahrnehmen. Die 
  finanzielle Unterstützung dieser Angebote muss sich auch nach diesem 
  Kriterium richten.
  
  \subsection{Integration}
  
  Augsburg liegt beim Bevölkerungsanteil mit Migrationshintergrund in der 
  Spitzengruppe deutscher Großstädte – bei den Kulturausgaben ist das nicht 
  zu erkennen. Ein jährlicher Volkstanzabend mit Ansprache eines Politikers 
  kann nicht als Kulturarbeit gelten – auch hier benötigt die Kommune eine 
  Bedarfsermittlung und engagiertere Unterstützung von Bürgerinitiativen.
  
  \subsection{Schaffung einer breit gefächerten Kulturszene}
  
  Nur über eine umfassende und inhaltlich breit aufgestellte Kulturpolitik 
  können Ausgaben gegenüber der Bevölkerung gerechtfertigt werden. Die 
  Piratenpartei Augsburg setzt sich einerseits dafür ein, vorhandene Gelder 
  gerechter für die einzelnen Wirkungskreise innerhalb der Stadt zu 
  verteilen. Etablierte Kulturinstitutionen sollen weiterhin gestützt werden, 
  aber nicht wie bisher zu Lasten kleinerer und neuer Initiativen. 
  Andererseits müssen langfristig mehr Gelder für den Kulturbereich 
  bereitgestellt werden, um seinen Erhalt in der drittgrößten Stadt Bayerns 
  zu gewährleisten. Hierfür gilt es in anderen Ressorts, soweit möglich, 
  Einsparungen vorzunehmen und kosteneffizient bereits bestehende Projekte zu 
  stärken, anstatt zusätzliche Parallelstrukturen zu schaffen. Initiativen 
  aus der Bevölkerung sind zu bevorzugen, da diese meist schon im Vorfeld 
  über breite Akzeptanz verfügen.
  
  \subsection{Raum für Kultur schaffen}
  
  Kulturschaffende brauchen genügend Platz, damit sie sich entfalten und 
  arbeiten können. Die Piratenpartei Augsburg fordert deshalb, dass 
  zusätzliche Räumlichkeiten für Kulturschaffende bereitgestellt werden. 
  Gerade ungenutzte Gebäude und Brachflächen im Stadtgebiet können für solche 
  Zwecke optimal genutzt werden. Die anerkennenswerte Arbeit der Kulturpark 
  West gGmbh verdient mehr Unterstützung, auch und gerade finanziell.
  
  Oftmals entwickeln sich rund um solche Objekte aus der Bevölkerung 
  bedarfsorientierte Initiativen. Diese gilt es von Seiten der Stadtregierung 
  nach Kräften zu unterstützen und die bürokratischen Hürden möglichst 
  niedrig zu halten. Allein schon die mediale Aufmerksamkeit für den 
  „Grandhotel Cosmopolis“ sollte Grund genug für eine Stadtregierung sein, 
  vergleichbare Initiativen zu fördern.
  
  Ergänzend zu diesen, meist langfristig angelegten Projekten, gilt es 
  Zwischennutzung zu ermöglichen: Gebäude und Flächen, auch wenn sie nur 
  wenige Monate leer stehen, können durch innovative Ideen eine große 
  Bereicherung eines gesamten Quartiers sein. Beste Beispiele sind hierfür 
  das „Jean Stein“ auf dem ehem. Hasenbräu-Gelände oder das „Muhackl oder 
  Blutwurst“ am Perlachberg. Eine Koordinationsstelle für Gebäude, Räume oder 
  Flächen zur kulturellen Zwischennutzung ist wünschenswert.
  
  \subsection{Kulturzonen}
  
  Rechtssicherheit für Bewohner und Kulturschaffende besteht derzeit nur 
  eingeschränkt. Gerade im Innenstadtbereich, aber auch im näheren Umkreis 
  von Kulturbetriebsstätten in anderen Bezirken herrscht Unsicherheit über 
  zulässige Lärmemissionswerte und Berwertung des Verkehrsaufkommens zu den 
  Kultureinrichtungen und -veranstaltungen. Hier kann die Stadtregierung bzw 
  der Stadtrat Abhilfe schaffen und durch Beschluss definieren, welche 
  Strassenzüge den bestehenden Charakter eines Misch- und welche tatsächlich 
  den eines auch von Verkehrsgeräusch (innenstadttypisch sind 70 dba tags und 
  60 dba nachts, bei Kopfsteinpflasterbelag höher) abgeschirmten reinen 
  Wohngebiets haben.
  
  \subsection{Einführung von Pauschaltickets}
  
  \subsubsection{Zoo-Ticket}
  
  Die Piratenpartei Augsburg plant die sinnvolle Nutzung der bestehenden 
  Park\&Ride-Plätze durch die Einführung eines sog. „Zoo-Tickets“. Das Ticket 
  kann an allen Fahrkartenautomaten erworben werden und ermöglicht dadurch im 
  Zoo vergünstigen Eintritt. Besucher können mit diesem Ticket für Bus und 
  Straßenbahn aus dem gesamten Tarifgebiet der Zone 10 und 20 bis zum Zoo 
  fahren. Dort gilt das Ticket direkt als Eintrittskarte, wodurch auch 
  Wartezeiten vermieden werden.
  
  \subsubsection{Garten-Ticket}
  
  Analog zum Zoo-Ticket soll ein Ticket für den Botanischen Garten inkl. 
  ÖPNV-Anfahrt geschaffen werden. Die Strukturierung mit Zone 10 und 20 sowie 
  dem vergünstigten Eintritt ist gleich zum Zoo-Ticket, nur der Preis 
  differiert, da der Botanische Garten an sich günstiger als der Zoo ist.
  
  Durch den Pauschalpreis der Ticket wird nicht nur die Attraktivität von Zoo 
  und Botanischem Garten gesteigert, sondern auch die der ÖPNV-Nutzung. 
  Zusätzlich wird - gerade an den besuchsstarken Wochenenden und Feiertagen - 
  der innerstädtische Bereich vom Individualverkehr entlastet.
  
  \subsection{Konkrete Forderungen der PIRATEN an die lokale Kulturpolitik}
  
  \subsubsection{Einführung eines Kulturtickets}
  
  Die zeitgemäße Vermittlung von musealen Objekten sollte nicht nur einer 
  einzigen Zielgruppe zugänglich gemacht werden, sondern vielen heterogenen 
  gesellschaftlichen Gruppierungen. Nicht nur Kinder brauchen einen 
  niedrigschwelligen Zugang zu musealer Bildung, auch Erwachsene sollen unter 
  dem Stichwort „lebenslanges Lernen“ zu neuen Impulsen und Denkmustern 
  angeregt werden. Eine radikal modernisierte Museumspädagogik kann helfen, 
  angestaubte Inhalte neu zu kontextuieren und sie größeren Publikumsgruppen 
  näher zu bringen. Dabei sollte man als wichtigen Leitgedanken anführen, 
  dass Kunst zu den Menschen, statt Menschen zur Kunst gebracht werden muss. 
  Dieser Leitgedanke soll durch ein sogenanntes Kulturticket in Augsburg für 
  jeden Bürger wieder erschwinglich und zugänglich gemacht werden. Dank der 
  Einsparungen bei den Zuschüssen für Hochkultur ist eine Umverteilung von 
  Finanzmitteln möglich und man erhält mit diesem Ticket in sämtlichen Museen 
  und öffentlichen Gebäuden vergünstigten Eintritt. Durch dieses Ticket soll 
  beim lokalen, aber auch überregionalen Publikum wieder das Interesse auf 
  Museen und Kunst geweckt werden. Es werden Barrieren abgebaut und 
  gleichzeitig soziale Hürden genommen, um Menschen mit geringem Einkommen 
  die Teilhabe an kulturellem (und damit sozialem) Leben zu ermöglichen.
  
  \subsubsection{Nachwuchsförderung im Kulturbereich}
  
  In der heutigen Zeit dürfen wir die Nachwuchsförderung im Kulturbereich 
  nicht vergessen. Dabei gehört nicht nur dazu, dass Heranwachsende 
  kulturelle Ereignisse miterleben, sondern auch die aktive Teilnahme ist von 
  größter Bedeutung. Um eine aktive Teilnahme an Kunst und Kultur zu 
  ermöglichen ist eine Vernetzung der Schulen mit Institutionen der 
  Soziokultur, der Laienkultur, sowie der in öffentlicher Hand befindlichen 
  Kulturbetriebe enorm wichtig. Die Piraten Augsburg fördern deshalb die 
  Vernetzung mit außerschulischen Institutionen, um den Kindern und 
  Jugendlichen eine aktive Teilnahme an Kunst und Kultur zu ermöglichen.
  
  Wir müssen Bildungs- und Kulturpolitik wieder stärker verzahnen, damit 
  Kultur allen näher gebracht werden kann.
  
  \subsubsection{Spezialfall Stadttheater}
  
  In Augsburg fliesst die Hälfte des städtischen Kulturetats in den Erhalt 
  und den Betrieb des Stadttheaters – ein klarer Hinweis darauf, dass sich 
  die 24st-grösste deutsche Stadt ein solches Prestigeobjekt nicht leisten 
  kann. Aus der chronischen Unterfinanzierung sowohl des Kultur- als auch des 
  Theateretats müssen daher Lösungswege führen. Um nicht etwa zu 
  privatwirtschaftlichen Lösungen (etwa Eintrittspreisstaffelung nach 
  Wohnort) greifen zu müssen, ist ein Umwidmen zum Bezirks- oder 
  Staatstheater mit entsprechender Unterstützung aus übergeordneten 
  Verwaltungsgliederungen vorzuziehen. Der Kulturetat muss die 
  Gesamtsituation der urbanen Kulturszene widerspiegeln, Theater haben 
  allerdings grundsätzlich einen gesellschaftlichen Auftrag zu erfüllen, 
  nicht nur einen historisch begründeten. Die Piratenpartei setzt sich 
  deshalb dafür ein, stärker auf projektgebundene Förderung zu setzen, um so 
  die Vielfalt der lokalen Kulturszene innerhalb des Theaters abzubilden und 
  durch das Theater zu fördern. Ohne ständige zusätzliche Mittel, etwa aus 
  dem Landeskulturetat, ist eine Erfüllung des kulturellen 
  Versorgungsauftrags durch das Theater auf Dauer nicht vorstellbar. 
  Andernfalls müssen weniger Produktionen und Premieren durchgeführt werden, 
  um zeitliche und organisatorische Freiräume für Projektkultur zu schaffen.
  
  \subsubsection{Spezialfall Museen}
  
  Die städtischen Museen beanspruchen einen signifikanten Teil der 
  städtischen Kulturmittel, ohne dass ein Bildungsauftrag sichtbar wäre. Hier 
  fehlt ein gesamtstädtisches Museumskonzept, das sich auf die Bedürfnisse 
  der Bevölkerung wie auch die Ausgabensituation im Kultursektor bezieht. Ein 
  Weg dorthin wäre das gezieltere Verteilen der musealen Ressourcen, etwa 
  ausgewähltere Öffnungszeiten im Rahmen eines Konzeptes für 
  selbstverständlichere Nutzung durch die Bürger.
  
  \subsubsection{Spezialfall Brechtfestival}
  
  Augsburgs Geschichte verfügt nur über wenige international bekannte Kultur- 
  und Wissensschaffende – ein Hinweis auf konsequent fehlende 
  Nachwuchsförderung. Das Brechtfestival könnte hier zur Belebung lokalen 
  Kulturschaffens und als geistiger Nährboden für künftige Brechts und 
  Mozarts dienen. Der bisher vor allem für Aussendarstellung genutzte und 
  leider nicht wirklich transparente Etat wäre im Sinne Brechts besser für 
  episches Volkstheater aus der Mitte der Stadtbevölkerung einzusetzen.
  
  \subsubsection{Spezialfall Popkultur}
  
  Die unglückliche Verquickung von Wahlkampf und Popkultur führte in Augsburg 
  zur Installation eines Popkulturbeauftragten, der wegen des politischen 
  Spagats von Anfang an ohne echte Chance auf Gestaltungserfolg war und in 
  dieser Situation letztlich auch resignierte. Popkultur ist grundsätzlich 
  nicht von Förderung abhängig, sondern auch im privatwirtschaftlichen Rahmen 
  überlebensfähig. Allerdings kommt der kommunalen Kulturpolitik auch hier 
  eine verantwortungsvolle Aufgabe zu, die der Nachwuchsförderung. Erprobte 
  Instrumente bilden hier die Schaffung bzw ausreichende Finanzierung von 
  Jugend- und Kulturzentren, um ein Gedeihen von Jugend- und Popkultur 
  bereits unterhalb der Wirtschaftlichkeitsschwelle zu ermöglichen. 
  
  \section{Portal / Datenbank für kostenlose Veranstaltungen in Augsburg}
  
  Die Stadt Augsburg soll, um die Attraktivität der Stadt zu steigern und 
  gesellschaftliche Teilhabe auch jenseits von Konsum zu fördern, ein Portal 
  betreiben, in dem Veranstalter von Vorträgen, Podiumsdiskussionen oder 
  ähnlichem, bei denen kein Eintritt verlangt wird, ihre Veranstaltungen 
  kostenfrei einstellen (lassen) können. Es ist momentan sehr schwer von 
  solchen Veranstaltungen Kenntnis zu nehmen, da nicht einmal die städtischen 
  Gebäude, in denen die Räumlichkeiten für solche öffentlichen 
  Veranstaltungen angemietet sind, auf deren Internetpräsenz darüber 
  informieren. Das ist schade für die Bürger, aber auch schade für die 
  Veranstalter die unentgeltlich einen Beitrag zum kulturellen Leben der 
  Stadt leisten wollen und sicher auch eine regere Teilnahme begrüßen würden. 
  Eventuell wäre mit einer Zunahme von Veranstaltungen dieser Art zu rechnen, 
  wenn sich heraus stellen würde, dass stets genug Interessenten kommen und 
  nicht nur die Personen, die sich sowieso im Dunstkreis der jeweiligen 
  Organisation bewegen, anwesend sind, damit der Raum nicht noch leerer ist. 

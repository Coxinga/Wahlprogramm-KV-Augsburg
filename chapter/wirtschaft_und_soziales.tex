\chapter{Wirtschaft und Soziales}

  \section{Abgeordnetenwatch für Augsburg Stadt und Landkreis}
  
  Die Piratenpartei Augsburg fordert die Einführung einer Kommunalinstanz von 
  Abgeordnetenwatch für die Stadt und den Landkreis. Nach Angaben von 
  Abgeordnetenwatch entstehen hierfür jeweils 100€ an Kosten pro Monat. Wir 
  sind der Meinung, dass dies von der öffentlichen Hand getragen werden kann, 
  bis sich genug Spender aus der Bürgerschaft finden. Die Möglichkeiten der 
  bürgerlichen Teilhabe an Entscheidungsprozessen bzw. deren 
  Nachvollziehbarkeit, sowie der Kontakt zu den gewählten Vertretern der 
  Bevölkerung, wird dadurch massiv gesteigert. 
  
  In vielen deutschen Kommunen wird dieses System bereits praktiziert, 
  darunter in Bayern Memmingen, München, Regensburg, Fürstenfeldbruck und der 
  Landkreis Coburg.\footnote{\url{http://www.abgeordnetenwatch.de/%
  kommunen-933-0.html}} Die Rückmeldungen aus den Kommunen ist sehr 
  positiv\footnote{\url{https://netzpolitik.org/2011/%
  abgeordnetenwatch-auf-kommunaler-ebene/}}
  
  \section{Digitaler Tourismusführer}
  
  Digitale Endgeräte fördern durch ihr Kartenmaterial und durch ihre 
  Internetfähigkeit die Möglichkeiten der Touristen zur unabhängigen Erkundung 
  des Stadtgebiets. Diesen Tourismus wollen wir Piraten fördern, da unserer 
  Meinung nach Touristen, die die Stadt auf eigene Faust erkunden, bereit 
  sind, mehr Zeit – und Geld – in der lokalen Wirtschaft und Gastronomie zu 
  lassen.
  
  Hierzu könnte man Nutzern des freien WLANs eine Startseite anbieten mit 
  einer Karte der Sehenswürdigkeiten, Geschäfte und gastronomischen Angeboten 
  – basierend auf dem freien Kartenangebot von Open Street Map. An den 
  touristischen Highlights könnten für Endgeräte lesbare Links (sogenannte 
  QR-Codes) angebracht sein, die eingescannt und auf textuelle und audiovisuellen 
  Inhalte leiten, welche weitere Informationen über den aktuellen 
  Aufenthaltsort bereitstellen. So erhalten die Touristen einen digitalen 
  Tourismusführer und ein einmaliges Erlebnis. Bei guter Aufbereitung kann man 
  die Touristen durch die gegebenen Informationen gezielt durch das 
  Weltkulturerbe steuern, ohne dass man ihnen die Freiheiten zur eigenen 
  Erkundung nimmt. In vielen anderen Städten, darunter Aalen[2], Eupen[3] und 
  Rottenburg[4] wird dieses Konzept bereits seit mehreren Jahren erfolgreich 
  praktiziert.
  
  \section{Mindestlohn Kommunaler Angestellter}
  
  Die Piratenpartei Augsburg fordert einen Mindestlohn von 8,50 EUR für alle 
  Angestellten und Mitarbeiter der Stadt Augsburg und ihrer Betriebe. Ebenso 
  fordern wir eine Auftragsvergabe durch die Stadt nur an Firmen, die diesen 
  Mindestlohn garantieren.
  
  Bei 173,3 Stunden im Monat sind das lediglich 1.473,05 EUR / Monat. (52 
  Wochen = 4,33 Wochen pro Monat Bei 40 Wochenstunden sind dies 173,3 Stunden 
  / Monat) Bei einem Alleinverdiener (25 Jahre, ges. KV/RV/ALV versichert, 
  Steuerklasse 1, keine Kinder) bleiben damit Netto 1.074,94 
  EUR.\footnote{\url{http://www.brutto-netto-rechner.info/gehalt/%
      gehaltsrechner-arbeitgeber.php}}
   
  
